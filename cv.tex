%-------------------------
% Auteur : Radhwen Saidi
%------------------------

\documentclass[10pt]{article}
\usepackage[empty]{fullpage}
\usepackage{titlesec}
\usepackage[usenames,dvipsnames]{color}
\usepackage{enumitem}
\usepackage[hidelinks]{hyperref}
\usepackage{fancyhdr}
\usepackage{multicol}
\usepackage{tabularx}
\usepackage{array}

\addtolength{\oddsidemargin}{-0.6in}
\addtolength{\evensidemargin}{-0.6in}
\addtolength{\textwidth}{1.2in}
\addtolength{\topmargin}{-0.8in}
\addtolength{\textheight}{1.65in}

\titleformat{\section}{
  \vspace{-6pt}\scshape\raggedright\large
}{}{0em}{}[\color{Black}\titlerule \vspace{-6pt}]

\newcommand{\resumeSubheading}[4]{
  \vspace{-4pt}\item
    \begin{tabularx}{\textwidth}[t]{X r}
      \textbf{#1} & \textit{\small#2} \\
      \textit{\small#3} & \textit{\small#4} \\
    \end{tabularx}\vspace{-2pt}
}

\newcommand{\resumeItem}[2]{\item\small{\textbf{#1 :} #2}}

\newcommand{\resumeKeywords}[1]{\item\small{\textbf{\textcolor{MidnightBlue}{Technologies :}} \textit{#1}}}

\newcommand{\resumeSubHeadingListStart}{\begin{itemize}[leftmargin=*]\setlength\itemsep{0pt}}
\newcommand{\resumeSubHeadingListEnd}{\end{itemize}}

\newcommand{\resumeItemListStart}{\begin{itemize}[leftmargin=*]\setlength\itemsep{0pt}}
\newcommand{\resumeItemListEnd}{\end{itemize}}

%-------------------------------------------
%%%%%%  CV STARTS HERE  %%%%%%%%%%%%%%%%%%%%%%%%%%%%
\begin{document}

%----------HEADING-----------------
\begin{tabularx}{\textwidth}{l@{\extracolsep{\fill}}r}
  \textbf{\LARGE Radhwen Saidi} & \href{mailto:radhwen.saidi@outlook.com}{radhwen.saidi@outlook.com} \\
  \textbf{\large \textcolor{MidnightBlue}{Data Scientist | Machine Learning Engineer}} & Mobile: +33 7 83 90 61 17 \\
  \href{https://www.linkedin.com/in/radhwen-saidi/}{LinkedIn: linkedin.com/in/radhwen-saidi} & Lille, France \\
\end{tabularx}

%-----------EDUCATION-----------------
\section{Formation}
  \resumeSubHeadingListStart
    \resumeSubheading
      {Mastère Spécialisé Expert En Sciences Des Données}{2022 -- 2023}
      {INSA Rouen Normandie}{Rouen, France}
    \resumeSubheading
      {Diplôme d'ingénieur en Informatique}{2020 -- 2022}
      {ENSIM (École Nationale Supérieure d'Ingénieurs du Mans)}{Le Mans, France}
  \resumeSubHeadingListEnd

%-----------EXPERIENCE-----------------
\section{Expériences Professionnelles}
  \resumeSubHeadingListStart

    \resumeSubheading
      {Data Scientist}{Depuis Oct. 2023}
      {ONEY BANQUE}{Lille, France}
      \resumeItemListStart
        \resumeItem{Modélisation Machine Learning}
          {Développement de solutions ML supervisées et non supervisées adaptées à divers cas métier (détection de fraude, recouvrement, segmentation client, scoring crédit), intégrant les dernières avancées technologiques.}
        \resumeItem{Développement d'Applications Analytiques}
          {Conception d'outils automatisés pour l'analyse des données, la génération de rapports détaillés, le monitoring des modèles, ainsi que la mise en place d'alertes proactives et de plans d'action correctifs.}
        \resumeItem{Amélioration continue}
          {Optimisation des modèles existants pour renforcer leur robustesse, performance et adaptabilité.}
        \resumeItem{Expertise ML référent}
          {Mise en œuvre de bonnes pratiques et guides méthodologiques, soutenus par une veille technologique, pour optimiser robustesse et alignement des filiales internationales.}
        \resumeKeywords{Python • R • SQL • Machine Learning • Databricks • MLflow • Evidently AI • PySpark • Snowflake}
      \resumeItemListEnd

    \resumeSubheading
      {Data Scientist}{Oct. 2022 -- Sept. 2023}
      {Crédit Agricole Consumer Finance}{Lille, France}
      \resumeItemListStart
        \resumeItem{Projet NLP -- Catégorisation de transactions bancaires}
          {Mise en place d’un pipeline complet : \textbf{Système d’annotation intelligent} (scores majoritaires, règles métier, gestion de priorités) associé à une \textit{analyse sémantique via t-SNE} pour détecter et corriger les incohérences, \textbf{prétraitement avancé} (Regex, corrections orthographiques) et \textbf{modèle BERT personnalisé} (fine-tuning sur corpus bancaire) déployé via un pipeline d’inférence scalable, assurant une catégorisation fiable et robuste.}
        \resumeKeywords{Python • Regex • Snorkel • Hugging Face • Transformers (BERT) • TensorFlow • t-SNE}
        \resumeItem{Projet Open Banking -- Système de scoring de crédit}
          {Intégration de données APIs et internes, optimisation avancée et sélection d’indicateurs clés ; conception d’un modèle XGBoost, suivi via MLflow et explicabilité stratégique avec SHAP.}
        \resumeKeywords{Python • Machine Learning • MLflow • SQL • APIs • SHAP}
      \resumeItemListEnd

    \resumeSubheading
      {Stage de fin d'études – Data Scientist/Machine Learning Engineer}{Mars 2022 -- Sept. 2022}
      {E-nno Switzerland SA}{Genève, Suisse}
      \resumeItemListStart
        \resumeItem{Détection d'anomalies énergétiques}
          {Développement d'un modèle innovant pour identifier les dysfonctionnements techniques et problèmes de qualité des données.}
        \resumeItem{Pipeline ETL et ingestion cloud}
          {Extraction, transformation et chargement des données issues des capteurs avec intégration sur AWS, optimisant ainsi les flux de données.}
        \resumeItem{Automatisation et déploiement}
          {Packaging via Poetry, déploiement automatisé (Docker, GitLab CI/CD) et orchestration avec Airflow et Kubernetes.}
        \resumeItem{Surveillance proactive}
          {Mise en place d'alertes et dashboards (Grafana) pour un monitoring continu des performances.}
        \resumeKeywords{Python • PostgreSQL • InfluxDB • Poetry • Airflow • Docker • AWS • CLI • GitLab • CI/CD • Kubernetes}
      \resumeItemListEnd

    \resumeSubheading
      {Stage Data Science}{Juil. 2021 -- Sept. 2021}
      {Teamwill Consulting}{Paris, France}
      \resumeItemListStart
        \resumeItem{Modélisation prédictive pour CrowdFunding}
          {Conception d'un modèle Machine Learning prédictif enrichi par une analyse NLP sur des tweets collectés via WebScraping, visant à prédire le succès ou l'échec d'un projet de crowdfunding.}
        \resumeItem{Chatbot intelligent}
          {Développement d'un chatbot conversationnel basé sur des réseaux de neurones, intégré à la plateforme pour fournir des informations sur les projets de crowdfunding.}
        \resumeItem{Développement et déploiement de l'application Flask}
          {Création d'une application web Flask intégrant le modèle prédictif, l'analyse de sentiment et le chatbot, suivie du déploiement sur AWS via EC2.}
        \resumeKeywords{Python • Scikit-Learn • Flask • WebScraping • SQL • AWS EC2 • NLP • Keras • Analyse de Sentiment • Chatbot}
      \resumeItemListEnd

  \resumeSubHeadingListEnd

%-----------TECHNICAL SKILLS-----------------
\section{Compétences Techniques}
\begin{itemize}[leftmargin=*]
  \item \textbf{Langages et Frameworks :} Python, R, Flask, Java, C\#.
  \item \textbf{Machine Learning et IA :} Scikit-learn, PyTorch, TensorFlow, NLP, LLMs (LangChain, LangGraph, RAG, Fine-tuning).
  \item \textbf{Big Data et Cloud :} Spark/PySpark, SQL/NoSQL, AWS (EC2, S3, ECR, SageMaker), Heroku, Snowflake, Databricks.
  \item \textbf{MLOps et Déploiement :} Docker, Airflow, MLflow, CI/CD, GitHub/GitLab, Streamlit, FastAPI, Poetry, Grafana.
\end{itemize}

%-----------LANGUES ET CERTIFICATIONS-----------------
\section{Langues et Certifications}
\begin{itemize}[leftmargin=*]
  \item \textbf{Langues :} Français (Courant), Anglais (Professionnel, TOEIC : 865), Allemand (A2).
  \item \textbf{Certifications :} Machine Learning Scientist (DataCamp), Deep Learning Specialization (Coursera), Project Management (IBMI).
\end{itemize}

\end{document}